\documentclass{article}
\usepackage[utf8]{inputenc}
\usepackage{amsmath}
\usepackage{amssymb}
\usepackage{color}
\usepackage{graphicx}
\usepackage{float}
\usepackage{physics}
\let\vec\mathbf
\title{main.tex }
\author{Henrik Linder}
\date{\today}
\begin{document}
\maketitle

\section{Introduction}


\section{Theory}
The Navier-Stokes equations are the fundamental equations of fluid mechanics. Using standard Einstein summation convention in cartesian coordinates, they are given as 
\begin{equation}
	\rho \pdv{u_{i}}{t} + \rho\left(u_{j}\pdv {u_{i}}{x_{j}}\right) = -\pdv{p}{x_{i}}+ \mu \pdv{x_{j}}\pdv{u_{i}}{x_{j}} + f_{i}, 
\end{equation}
with $u$ being the velocity, $x$ is position, $p$ is pressure, and $f$ is external forces on the body. 


So then , the averga of the momentum equations become 
\begin{equation}
	\begin{split}
		\rho \left(\pdv{U_{i}}{t}+U_{j}\pdv{U_{i}}{x_{j}}\right) &= -\pdv{P}{x_{i}} + \pdv{T_{ij}}{x_{j}} + \pdv{R_{ij}}{x_{j}} + \langle f_{i}\rangle \\&= \pdv{x_{j}}\left(-p\delta_{ij} + T_{ij} + R_{ij}\right) + \langle f_{i}\rangle
	\end{split}
\end{equation}

with 
\begin{equation}
	T_{ij} = \mu \left(\pdv{U_{i}}{x_{j}} + \pdv{U_{j}}{x_{i}}\right)
\end{equation}
\begin{equation}
	R_{ij} \equiv -\langle \rho u_{i}'u_{j}'\rangle.
\end{equation}







\end{document}
