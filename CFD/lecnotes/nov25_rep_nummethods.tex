\documentclass{article}
\usepackage[utf8]{inputenc}
\usepackage{amsmath}
\usepackage{amssymb}
\usepackage{color}
\usepackage{graphicx}
\usepackage{float}
\usepackage{physics}
\title{nov25 rep nummethods.tex }
\author{Henrik Linder}
\date{\today}
\begin{document}
\maketitle

There are some older methods that use the vorticity equation rather than NS
then the fluid velocity is represented bu 3 prts:	
	free stream contribution
	solid body contribution
	vortex-vortex contribution
They noted that few particles gave a much better result. The problem was the center singularity in the potential vortex solution that they used. Solution: use vortex woth center core. (remove the singularity) Then it worked.

\section{Finite element method:} 
the method used in comsol (not that common in CFD software, ususlly FVM)
Rectangles at border, because vertical resolution is moreimportant than horizontal (vel gradient larger vertically)

FEM recipe 
\begin{equation}
	\nabla\cdot(k\nabla T) = g(T,x)
\end{equation}

Degrees of Freedom 
nrnodes*nrvariables
nonodes = elementfactor*nrelements

so it depends on geometry, dimensions, order of approx., etc






\end{document}
